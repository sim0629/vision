\documentclass[a4paper,10pt]{article}

\usepackage{datetime}
\usepackage{fullpage}
\usepackage{indentfirst}
\usepackage{amsmath}
\usepackage{amsfonts}
\usepackage{amssymb}
\usepackage{bm}
\usepackage{enumerate}
\usepackage{listings}
\usepackage{graphicx}
\usepackage{float}
\usepackage{multirow}

\linespread{1.5}

\begin{document}
\title{Answers to the theory questions - Homework 4 \\
  \large Computer Vision (2016 Spring)}
\author{2009-11744 Gyumin Sim}
\maketitle

\section*{Question 1}

\begin{table}[H]
\centering
\begin{tabular}{|c|c|c|}
\multicolumn{3}{c}{$I(t)$} \\
\hline
1 & 1 & 1 \\
\hline
1 & 1 & 1 \\
\hline
1 & 1 & 1 \\
\hline
\end{tabular}
\quad
\begin{tabular}{|c|c|c|}
\multicolumn{3}{c}{$I(t + \delta t)$} \\
\hline
1 & 1 & 1 \\
\hline
1 & 1 & 1 \\
\hline
1 & 1 & 1 \\
\hline
\end{tabular}
\end{table}

Either horizontal or vertical components could not be recovered
when all pixels of the image has identical color (intensity) like the example above.
Using area-based method to get optical flow,
all derivatives ($I_x$, $I_y$, and $I_t$) of all pixel positions will be zero whatever the motion is.
Therefore, the $2 \times 2$ matrix on the left-hand side of the equation below is not invertible,
so $u$ and $v$ would be unknown.

\begin{align*}
\begin{bmatrix}
\sum\nolimits I_x^2 & \sum\nolimits I_x I_y \\
\sum\nolimits I_y I_x & \sum\nolimits I_y^2
\end{bmatrix}
\begin{bmatrix}
u \\
v
\end{bmatrix}
=
\begin{bmatrix}
- \sum\nolimits I_x I_t \\
- \sum\nolimits I_y I_t
\end{bmatrix}
\end{align*}

\section*{Question 2}

\begin{table}[H]
\centering
\begin{tabular}{|c|c|c|}
\multicolumn{3}{c}{$I(t)$} \\
\hline
0 & 0 & 0 \\
\hline
0 & 1 & 0 \\
\hline
0 & 0 & 0 \\
\hline
\end{tabular}
\quad
\begin{tabular}{|c|c|c|}
\multicolumn{3}{c}{$I(t + \delta t)$} \\
\hline
0 & 0 & 0 \\
\hline
0 & 0 & 0 \\
\hline
0 & 0 & 1 \\
\hline
\end{tabular}
\end{table}

Both horizontal and vertical components of motion could be always recovered
when only one pixel has different color (intensity) like the example above.
The $u$ and $v$ could be known by using area-based method.

\section*{Question 3}

The optical flow equation still hold
if the motion of the camera doesn't violate the assumption of small motion.
To figure out moving objects,
get the dominant motion by computing the optical flow of the entire scene,
and then subtract the dominant motion from the scene
to filter out the optical flow caused by the camera motion.

\section*{Question 4}

Let the coordinate of a point in the world be $(x_0, y_0, z_0)$,
the velocity $V$ be $(u, v, w)$, and the focal length be $1$.
Then, the projection point on the image plane would be
$(x', y') = ((x_0 + ut) / (z_0 + wt), (y_0 + vt) / (z_0 + wt))$ on a time $t$.
When $t$ goes to the negative infinity to get the origin of the optical flow vector,
all points converge on the same point $(u/w, v/w)$, which is called the focus of expansion.

\end{document}
