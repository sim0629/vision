\documentclass[a4paper,10pt]{article}

\usepackage{datetime}
\usepackage{fullpage}
\usepackage{indentfirst}
\usepackage{amsmath}
\usepackage{amsfonts}
\usepackage{amssymb}
\usepackage{bm}
\usepackage{enumerate}
\usepackage{listings}
\usepackage{graphicx}
\usepackage{float}
\usepackage{multirow}

\linespread{1.5}

\begin{document}
\title{Answers to the theory questions - Homework 5 \\
  \large Computer Vision (2016 Spring)}
\author{2009-11744 Gyumin Sim}
\maketitle

\section*{Question 1}

AlexNet consists of 8 layers.
The last 8th layer's output has 1,000 neurons.
So, I assume the vector referred to the question is the output of the 7th layer, which has 4,096 neurons.
Each layer's output is calculated as follows:
\begin{enumerate}
\item From $224 \times 224 \times 3$ input image, pad $3$ pixels to width and height to get $227 \times 227 \times 3$ vector.
Convolve it with $96$ kernels of size $11 \times 11 \times 3$ with a stride of $4$ pixels to get $55 (= (227-11)/4+1) \times 55 \times 96$ output.
\item Split the input into two of size $55 \times 55 \times 48$ and process each on two different GPUs.
On each GPU, pool (max pooling) each input with a size of $3 \times 3$ with a stride of $2$ pixels to get $27 (= (55-3)/2+1) \times 27 \times 48$ vector.
Convolve it with $128$ kernels of size $5 \times 5 \times 48$ to get $27 \times 27 \times 128$ output for each GPU.
\item For each GPU, pool each input as the same way to the previous layer to get $13 (= (27-3)/2+1) \times 13 \times 128$ vector.
Merge the vectors for two GPUs to get $13 \times 13 \times 256$ vector.
Convolve it with $384$ kernels of size $3 \times 3 \times 256$ to get $13 \times 13 \times 384$ output.
\item Split the input into two of size $13 \times 13 \times 192$ and process each on two different GPUs.
On each GPU, convolve each input with $192$ kernels of size $3 \times 3 \times 192$ to get $13 \times 13 \times 192$ output.
\item On each GPU, convolve each input with $128$ kernels of size $3 \times 3 \times 192$ to get $13 \times 13 \times 128$ output.
\item Merge two inputs for each GPU to get $13 \times 13 \times 256$ vector.
Pool it as the same way to the previous layer to get $6 (= (13-3)/2+1) \times 6 \times 256$ vector.
With a fully-connected layer, calculate 4,096 dimensional vector.
\item With another fully-connected layer, calculate another 4,096 dimensional vector.
\end{enumerate}

\section*{Question 2}

Pooling outputs a value for a given patch using corresponding methods.
Max pooling selects the maximum value in the patch,
min pooling selects the minimum value in the patch,
and mean pooling calculates the average value of all values in the patch.

\end{document}
