\documentclass[a4paper,10pt]{article}

\usepackage{datetime}
\usepackage{fullpage}
\usepackage{indentfirst}
\usepackage{amsmath}
\usepackage{amsfonts}
\usepackage{amssymb}
\usepackage{bm}
\usepackage{enumerate}
\usepackage{listings}

\linespread{1.5}

\begin{document}
\title{Answers to the theory questions - Homework 1 \\
  \large Computer Vision (2016 Spring)}
\author{2009-11744 Gyumin Sim}
\maketitle

\section*{Question 1}

\subsection*{(a)}

$f$ is unit impulse function or Dirac delta function $\delta(x)$.
\begin{align*}
\delta(x) = \begin{cases} +\infty, & x = 0 \\ 0, & x \ne 0 \end{cases} where \int_{-\infty}^\infty \delta(x) \, dx = 1.
\end{align*}

In cases of discrete functions, $f$ is Kronecker delta function $\delta[n]$.
\begin{align*}
\delta[n] = \begin{cases} 0, & n \ne 0 \\ 1, & n = 0.\end{cases}
\end{align*}

\subsection*{(b)}

At a point $(a, b)$, the direction of the normal of the dominant edge is
\begin{align*}
\theta(a, b) = \arctan(I_y(a, b) / I_x(a, b)).
\end{align*}

Let $(u, v)$-axis be the rotated axis of $(x, y)$-axis by the angle of $\theta$.
That is
\begin{align*}
\begin{pmatrix}
u \\
v
\end{pmatrix} =
\begin{pmatrix}
  \cos \theta & \sin \theta \\
- \sin \theta & \cos \theta
\end{pmatrix}
\begin{pmatrix}
x \\
y
\end{pmatrix}.
\end{align*}

To smooth $I(a, b)$ by $\sigma_1$ in the direction of the dominant edge and by $\sigma_2$ in the perpendicular direction,
we can apply a Gaussian kernel to $I(a, b)$ along the $u$-axis by $\sigma_2$ and along the $v$-axis by $\sigma_1$.
So,
\begin{align*}
I'(a, b) &= I(x, y) * g(u, v) \\
&= \iint I(a - x, b - y) \frac{1}{2\pi \sigma_1 \sigma_2} \exp \left( -\frac{1}{2}\left(\frac{u^2}{\sigma_2^2} + \frac{v^2}{\sigma_1^2}\right) \right) dx dy \\
&= \iint I(a - x, b - y) \frac{1}{2\pi \sigma_1 \sigma_2} \exp \left( -\frac{1}{2}\left(\frac{(x \cos \theta + y \sin \theta)^2}{\sigma_2^2} + \frac{(-x \sin \theta + y \cos \theta)^2}{\sigma_1^2}\right) \right) dx dy .
\end{align*}

\section*{Question 2}

\subsection*{(a)}

Circles in 2-D space can be identified by three elements $(x, y, r)$ where $(x, y)$ is the coordinate of the center and $r$ is the radius.
A point $(a, b)$ can be transformed to Hough space $(x, y, r)$ as,
\begin{align*}
\sqrt{ (x - a)^2 + (y - b)^2 } = r
\end{align*}
which represents a set of all circles that pass the point $(a, b)$.
It will be a conical surface.

\end{document}
