\documentclass[a4paper,10pt]{article}

\usepackage{datetime}
\usepackage{fullpage}
\usepackage{indentfirst}
\usepackage{amsmath}
\usepackage{amsfonts}
\usepackage{amssymb}
\usepackage{bm}
\usepackage{enumerate}
\usepackage{listings}
\usepackage{graphicx}
\usepackage{float}

\linespread{1.5}

\begin{document}
\title{Writeup - Homework 2 \\
  \large Computer Vision (2016 Spring)}
\author{2009-11744 Gyumin Sim}
\maketitle

\section*{Q1}

Orientation-consistency cue is visual properties that two different points on a surface with the same orientation look similar.
The surface normal of each point, which is essential for the shape reconstruction, can be inferred from orientation-consistency by taking a image with an additional reference object whose normal is known like a sphere or a cylinder and by matching each point of a target to the most similar point of the reference object.

\section*{Q2}

To reconstruct a shape with orientation-consistency cue, we should compare each point of the target object with each point of the reference object and identify the best matching point, but there may be a lot of similar appearance (but different orientation) of points on just one illumination condition.
Many images on different illuminations or light sources make the points distinguishable, so they can improve the quality of the recovered shape.

\section*{Q3}

First, I precomputed the normal vectors for the reference object with the images masked out the background.
Second, I created object vectors for both of the target object and the reference object by concatenating object vectors of each RGB channel.
Finally, I got the normal vectors by gathering the precomputed normal vectors corresponding to the index returned by `kdtreeidx' function passed the target and reference object vectors.

\end{document}
